\documentclass[11pt,letter]{article}
\usepackage[top=1.00in, bottom=1.0in, left=1.1in, right=1.1in]{geometry}
\renewcommand{\baselinestretch}{1}
\usepackage{graphicx}
\usepackage{natbib}
\usepackage{amsmath}
\usepackage{hyperref}

\def\labelitemi{--}
\parindent=0pt
\parskip=5pt

\begin{document}
\bibliographystyle{/Users/Lizzie/Documents/EndnoteRelated/Bibtex/styles/besjournals}
\renewcommand{\refname}{\CHead{}}

\title{Climate Hazards \\ Log of meetings \& decisions}
\author{Lizzie, Isabelle Chuine, Ben Cook, Victor van der Meersch}
\date{\today}
\maketitle
\tableofcontents

% \section*{Some current issues}

\section{Meeting notes}


\subsection{22 juin, Stuff I did today and chatting with Victor}

Using Victor's nice file, I updated my code the PET is adjusted depending on temperature. On quick glance, it did not do much.

I tested whether increasing SD in a simple threshold model should make the leafout date earlier. Answer is no. 

\emph{Victor}:
PET should effect FruitIndex (and it does! Phew). It should not effect DroughIndex in PHENOFIT4 (also appears true). I compared between the PHENOFIT output before fixing PET and after. \\
It looks like the fixed dates for PHENOFIT must be the same date for all years, to discuss with Isabelle.\\

\subsection{20 juin, with Victor \& Isabelle}

Maybe don't change the WHC, if we are interested better to change the precipitation regimes.... 

\subsection{19 juin, with Victor \& Isabelle; email from Ben}

{\bf I am back in Montpellier!} Lots of stuff happening ....

Start year in phenofit (should be the first year you have data -- it will just be that you only get output from the FOLLOWING year). I already udpdated this in howto\_phenofit.txt.

\emph{Looking at historical results with Isabelle}
\begin{itemize}
\item Prunus has a stronger chilling limit, which may make the leaf unfolding dates later and mean that rarely is there frost.
\item The 0 (zero) FruitIndex for Fagus probably due to 0 LeafIndex (which is probably due to frost)
\item For Quercus there is -999 for FruitMaturationDate (only twice); Isabelle looked this up and it means the data is not set. For now I set the code to replace the -999 with 366. 
\end{itemize}

Check the frost resistance of the 3 spp (I added to my \_dothis list). 

\emph{Email from Ben (19 June 2023)}

\begin{quote}
For this we need to finalize which forecasted SSP scenarios to look at. We don't have many choices based on where our co-author Victor van der Meersch pulls data -- either SSP 245 or SSP585 (aka SSP 2 (4.5) etc. I believe). I said we should do SSP585 as that seems more likely than SSP245 right now. Questions:
\end{quote}

Okay, so SSP2-4.5 is probably the closest track to where we are heading thanks to the Paris Agreement. 8.5 is going to be VERY warm, and likely represent a worst case scenario that we are likely to avoid assuming current policies continue into the future. This is actually a pretty useful website that gets updated regularly, showing were we are likely to be based on different policy assumptions:

\url{https://climateactiontracker.org/global/cat-thermometer/}

So you can see, the blue range is centered on 2.7 C, which represents current policies that have been implemented. If I recall correctly, I believe multi-model for 4.5 puts us around 3 C, or a little higher, and multi-model for 8.5 put us around 5 C. So I think to be a bit easier to defend, 4.5 is the way to go. That said, if you want a range, I think it is fine to use and present both. It would be something similar to what we did for Nacho’s paper, where we compared 2 vs 4 degrees of warming.

\begin{quote}
In related news I don't think we ever confirmed which PET method to use. We cannot get the RH to do the easier one so are planning to use the Penman Monteith  ...

d) Does Penman Monteith sound okay to use for this?
\end{quote}

Yes, if you have the data for it, Penman-Monteith is considered the gold standard for PET estimates.

Sorry for the rush-the next few weeks are hectic, especially with Luca’s school year ending and holiday coming up. But things will slow down starting mid-July.

\subsection{16 juin, email from Victor with the excellent title, `PHENOFIT deligths'}

Ouch, this one was tricky.

I downloaded your files on my computer and tried running Phenofit on my side.
I didn’t find the error right away - even by adding several check messages printed to the log.

Drum roll... we must always have 4 comment lines before climate data, as in the historical climate files I sent you.
So far, I was lucky enough to avoid this error... 

\subsection{20 avril, with Isabelle \& I\~naki}
See green and gray notebook. 

\subsection{18-19 avril, discussions with I\~naki}

See also \verb|ChatsInaki.txt|. But here are the main notes copied over:
\begin{itemize}
\item References on damage due to high heat: 
\begin{itemize}
\item Thierry Simonneau worked on 2019 heat wave and why it destroyed so many vineyards in Montpellier (LaCave) -- he sent the presentation and thought they had a presentation about sentinel tool (more on anthocyanin) -- see \href{https://www6.inrae.fr/laccave/content/download/3466/35052/version/2/file/Canicule_Simonneau_25112021.pdf}{INRAe talk here}
\item Otherwise work may be in gray literature; he thinks in fruit trees (e.g., cherries get double fruits)
\item Wheat they think 25 C affects yield -- is that damage? If yes, they this in their models. 
\item  Apples with burned skin etc (also damage) and depends on canopy. 
\item Also check Zaka thesis and look up Gaetan (umlaut on e) Louarn (Zaka PhD advisor) ... look up P3F system they have and SICLEX
\end{itemize}
\item Incl. air temperature of 30-35 C means leaf surface temperature approaches protein denaturation limits; he does not know (ask Isabelle again)
\end{itemize}


\subsection{14 avril, with Isabelle \& Victor: PET}

PET ... we do not have RH from ERA5LAND so we cannot use Ben's easy equation (see below). Instead we could use the complicated Penman Monteith which is the \href{https://www.fao.org/3/x0490e/x0490e06.htm#TopOfPage}{Standard FAO method; $ET_o$ in Chapter 2}. See also \verb|docs/2023Apr_ETVictor.pdf.| And he shared his code.\\

$T_{dew}$ is $e_a$\\
$e_a$ is basically $vabar$\\
$e_s$ is basically $vas$ (and you can compute with $T_{min}$ and $T_{max}$ ...\\

\subsection{14 avril, with Isabelle}

\emph{This meeting covered:}
\begin{itemize}
\item References on risk (damage) versus longer growing season
\item What output from PHENOFIT to look at (long discussion)
\item Visiting in June: Avoid 26-28 (away) June and 13, 15 (she is busy)
\end{itemize}

{\bf References on risk (damage) versus longer growing season}\\

Yann has some work (I think she meant frost damage); perhaps do a lit review for Chinese species. I asked about \emph{high temperatures} and she is not aware of any, but said there is some literature on floral necrosis (during spring) in crop literature probably. Also, there was heat damage at Peuchabon (field station) due to high temperature and some work on cuticles from this -- some by Nicolas Martin (who is now at INRAe Avignon, forest group) but she does not think that it is published. % https://www6.paca.inrae.fr/ecologie_des_forets_mediterraneennes/Les-personnes/Personnels-permanents/MARTIN-Nicolas

Air temperature of 30-35 C means leaf surface temperature approaches protein denaturation limits. See paper by Yann Vitasse on this, also ag literature

{\bf Which output from PHENOFIT?}\\

\emph{A little more complicated than I thought!}

\begin{enumerate}
\item LeafIndex -- impacted only by frost (and contributes to maturation index). We think it should be only impacted by frost. This is continually updated inside the model so Isabelle is checking when the version that is written out is taken. 
\item FruitIndex -- impacted only by frost (lost due to frost on flowers or perhaps small fruit present early enough to get frosted)
\item MaturationIndex -- this one is more of a pain than I realized! 
\begin{enumerate}
\item It depends on: Length of the growing season (including drought during it I think) and LeafIndex
\item So if we want to look at impacts of GSL we'd need to look at MaturationIndex x GSL and see how strong the relationship is (could also look MaturationIndex x LeafIndex) ... 
\end{enumerate}
\end{enumerate}

So we need to extract: 

\begin{enumerate}
\item LeafIndex
\item FruitIndex 
\item MaturationIndex
\item Fitness -- which is the product of survival*FruitIndex*MaturationIndex (in our case, survival should generally always be 1 since this is only below 1 due to lethal cold temperatures and we're not in that range and drought)
\item LeafSenescence and LeafUnfolding dates -- this will give GSL, which matters to MaturationIndex for QUEROB and FAGSYL (but likely not PINSYSL) ... could check
\item Could also check (not critical) that LeafUnfolding date is good metric -- it should be HIGHLY correlated with FloweringDate and FruitInitDate. 
\end{enumerate}

\emph{Pinus} spp mature fruit over two years; in PHENOFIT the maturation takes place in one year so flowering date and season length should matter less to MaturationIndex (may want to double check model). Thus we {\bf predict} that warming does not impact MaturationIndex for PINSYL as much as for other two species, but all will be impacted by increased variability due to frost loss (though now that I transcribe these notes, I think the least impact should be on species that leaf out the latest, which I suspect is also PINSYL).  \\

I asked about partitioning PHENOFIT fitness due to each part and Isabelle said they did something a little like this in Morin et al. 2006 `Process-based modeling of species’ distributions: What limits temperate tree species’ range boundaries' \\% https://esajournals.onlinelibrary.wiley.com/doi/pdfdirect/10.1890/06-1591.1

A few more notes maybe in green and gray notebook, but I think I got most stuff here. 

\subsection{7 avril, with Isabelle}

\emph{This meeting covered:}
\begin{itemize}
\item WHC (water holding capacity)
\item Coding simulations in PHENOFIT
\item Problems with fitness values
\end{itemize}

{\bf WHC.} Try 50 and 100\%. See `Lien entre dommages dus a  la crise et reserve utile en eau des sols.doc' on Lizzie's computer for some background that helped Isabelle select these. 

{\bf Coding simulations in PHENOFIT.} Use different longitudes as way to code different simulations. (Latitude is used in the model for photoperiod, but longitude is just used as location marker so should be fine.) Isabelle also thought same lat/lon with different data in different rows would work, but using longitude seems easier. 

{\bf Problems with fitness values.} I ran PHENOFIT on the historical data using my gradient of sites from Italy, through Germany and onward to Sweden, but I got pretty low fitness values for FAGSYL and QUEROB. Isabelle said this happens. They usually convert to $0/1$ so you don't notice this. She also said they have seen that the fitness is higher closer to where the data used to parameterize the model comes from  (see \verb|Phenofit4_Anne_Duputie.pdf| on Lizzie's computer). 

So, Isabelle said it might have to do with where the data that the parameterized with from (France) versus where our points are (more Central Europe). We discussed changing to other parameters for the leaf phenology (from a K. Kramer book), but then Isabelle realized that we don't have similar parameters to use for flowering. 

So I changed to sites that track more through France. \\


\subsection{4 April 2023 with Ben}\\

Calculating ET ... and a little more on simulated climate data 

Detrending removes trends in central tendency 
\begin{enumerate}
\item Use detrended data for long-term baseline
\begin{enumerate}
\item Z-score them
\item Then multiple the SD by 1.2 for a 20\% increase in variability *normal distribution 
\item Check your output 
\end{enumerate}
\end{enumerate}

ET ... {\bf MAKE sure to use min and max RH correctly}
\begin{itemize}
	\item if you have relative humidity and temperature, then that's enough 
	\item Use this: \url{https://en.wikipedia.org/wiki/Clausius%E2%80%93Clapeyron_relation#August%E2%80%93Roche%E2%80%93Magnus_formula}
	\item This is how Ben typically does it
	\item Need: max RH and min RH (and temperature)
	\item Remember: Tmax is associated Tmin RH (and vice versa)
	\item RH tells you how saturated the air is; use temperature to calculate how much water the air is
	\begin{itemize}
	\item Step 1: Put in the August Roche Magnus -- hypothetical maximum ($e_s$) using just $T$ -- see the link: Basically it's $6.1 ^{((17.625*T)/(T+234))}$
\item Step 2: Now multiple $e_s$ by RH: $e_S*(RH)$
\item (Make relative RH is measured as a fraction)
\item Now you have the ET!
\item Check with an online calculator! Or check that every degree warmer get about a 7\% increase. 
	\end{itemize}
	\end{itemize}

\subsection{19 mars 2023: Leaf index}\\

Isabelle wrote: I found LeafIndex : it is the leafIndex at the end of the season, the day before senescence date, so it integrates all frost damages since bud set (or almost).

\subsection{15 mars 2023}\\

See \verb|ChatsBenNotes.txt| for thoughts on how to do the climate.\\

Isabelle also emailed me:\\
I asked about a copy of Budyko, M. I. (1974) Climate and life. International Geographic Series (ed. by D.H. Miller), pp 508. Academic Press, New York, NY and she replied: "If I do, this is a paper copy which is somewhere in the furniture between our desks!"\\

PHENOFIT 5 reference for carbon model: A generic process-based SImulator for meditERRanean landscApes SIERRA): design and validation exercises (Mouillot)\\

Recommended refs in Chuine\_PTB2010 for why growing season length and growth might be related and wrote, "I haven’t done the biblio on that topic recently but you will find a couple of references in the second paper attached. As I said, growing season length is related to NPP but not necessarily to growth."

\subsection{17 mars 2023}\\

MW Isabelle:\\
- Two ideas for climate\\
- How to adjust for other climate variables\\
- How to ask for climate data from Victor \\
- Which months around budburst \\

\subsection{9 mars 2023}

\begin{enumerate}
\item Sites
\begin{enumerate}
\item Previously discussed: Southern beech forest as site (Massan and sites in Pyrenees) may not be ideal as lengthening of growing season is not happening in southern range (because drought is shortening the length of the growing season; senescence now in August sometimes). That's the REAL world though, the current PhenoFit shows it lengthening (Isabelle has a postdoc starting in May who may try to work on the drought aspect to fix this)
\item Right now the model (Phenofit 4; Delpierre) does not include drought effect on senescence ... just temperature and daylength affect senescence
\item But do we need specific sites? Only if we want to compare to what is happening.... But we do need for some climate data. \item But be sure to pick a site that is FLAT (because using climate re-analysis) 
\end{enumerate}
\item Adding heat damage: Need good experimental data ... 60C at surface for adult leaves, lower for young leaves (June 2019 saw leaf damage, even on Holm Oak) -- so no, don't add. 
\item Species 
\begin{enumerate}
\item There are parameterized models for 20 species or so, could use one of these. Or, an artificial species -- make up parameters. 
\item Work with a broadleaf and an evergreen needleleaf. 
\item Quercus robur and patraea are basically the same species. 
\item Actually do these three (based on comparisons and having well parameterized models for them) Do querob/quepet (pick one), fagsyl, and pinsyl. Look at the overlay distributions for somewhere flat and not too far south. Look at the new European atlas (European atlas of forest trees) .... \url{https://forest.jrc.ec.europa.eu/en/european-atlas/} .. see also SI of Duputie...
\end{enumerate}
\item Phenofit5 is still being parameterizing and tested. Used already for beach and Oak. 
\item What we'll measure? 
\begin{enumerate}
\item Phenofit 4 has annual fitness: sum(survival + reproductive success) ... reproductive success is based on ripe fruit by end of season before leaf senescence. Survival has a crude carbon metric. -- so fitness is constrained by spring frost and length of season. 
\item Survival is the product of 3 survival parts (temp, drought, carbon: most of survival is drought).
\item MaturationIndex is a metric of if the season was long enough .... so this is a way to measure season length
\item FruitIndex and LeafIndex only go below 1 due to frost ... so these are ways to measure frost loss. 
\item LeafDormancyBreakDate relates to leaf-unfolding and to frost risk. 
\item We can hold phenology constant (at least in PHENOFIT 4)...YES! Can do this -- when initializing you can do this, 'activate the date files' -- give it specific dates (see methods Duputie et al. 2015 on plasticity GCB). Not sure if it works for Phenofit 5 (see Gauzere Evolution Letters 2020 .... new standing variability in phenology may make this hard). 
\end{enumerate}
\item Some notes on climate. ... 
\begin{enumerate}
\item ERA5land model at 10km (includes scenarios and future) % https://cds.climate.copernicus.eu/cdsapp#!/dataset/reanalysis-era5-land?tab=overview
\item WHC is hard to calculate -- so many want to try across three values (low/medium/high). 
\end{enumerate}
\end{enumerate}

\subsection{Before 9 mars 2023}

Notes from getting going ... 

\begin{itemize}
\item What site and species do we want to use?
\begin{itemize}
\item Ideally one with an existing model we do not have to spend a long time verifying? 
\item Both historical climate and future climate ... so we can compare
\item But also one where some events (frost, heat damage) may have shifted at the same time that growing season length or seasonal warmth has had an impact.
\end{itemize}
\item High temperatures?
\item What do we want to measure?
\begin{itemize}
\item Phenological shifts related to end/start of season and particular damage events
\item Tissue or ideally reproduction lost due to each event type
\item Reproduction changes due to longer/warmer season
\item What else?
\end{itemize}
\item What \emph{in silica} experiments do we want to run? 
\begin{itemize}
\item Historical climate
\item CMIP future climate scenario (maybe) -- characterize future climate: what is mean shift, what is variance shift? 
\item Historical climate + 2C?
\item Historical climate with increased variability in spring (and summer?)
\item Historical climate + 2C? with increased variability in spring (and summer?)
\item We want to test how phenological shifts x extremes matter, so I wonder if there is a scenario where we can try to hold phenology more constant and layer on climate variability? Not sure what that will show us ... 
\end{itemize}
\end{itemize}

\end{document}
