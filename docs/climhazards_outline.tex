\documentclass[11pt,letter]{article}
\usepackage[top=1.00in, bottom=1.0in, left=1.1in, right=1.1in]{geometry}
\renewcommand{\baselinestretch}{1}
\usepackage{graphicx}
\usepackage{natbib}
\usepackage{amsmath}

\def\labelitemi{--}
\parindent=0pt
\parskip=5pt

\begin{document}
\bibliographystyle{/Users/Lizzie/Documents/EndnoteRelated/Bibtex/styles/besjournals}
\renewcommand{\refname}{\CHead{}}

\title{Climate Hazards: Outline} % Definition of hazards from survival model angle is the probability of an event (this seems perfect)
\author{Lizzie, Isabelle Chuine, Ben Cook, Victor van der Meersch} % Ask for a review by Frederik 
\date{\today}
\maketitle

{\bf Questions for co-authors...}
\begin{enumerate}
\item Only max temperature matters to heat extremes -- unlike hardiness where plants gain/lose tolerance, this doesn't really happen for heat extremes, right?
\item What extremes to think about? Heat, cold, drought ... ?
\item Is there any literature on heat damage as limiting seasons? Probably no; always drought. 
\item Any ideas on figures?
\end{enumerate}

\section{Outline}

\begin{enumerate}
\item Introduction: mean versus variance in climate change biology
\begin{enumerate}
\item Increasing interest in impacts of extremes
\item Lots on thermal curves and variance, including lots on Jensen's inequality
\item Lots of interest in shifts in variance from theoretical perspective, but often not linked to empirical reality
\item Most find increasing variance will reduce fitness (e.g., Vasseur), but how accurate is this?
\item As variance is a fundamental part of life history theory and physiology, it is likely more complicated
\item Here we ... 
\end{enumerate}
\item Why extremes matter
\begin{enumerate}
\item Overview of basics 
\begin{enumerate}
\item Lots on mean versus variance theoretically in life history theory (colors of noise etc.) 
\item And in physiology: mean development, but also how extremes limit distributions 
\item Molecular studies back up this complexity: variability versus mean temperatures underlie pathways to some events
\item But climate change smears across these two separate approaches as we need to predict outcomes across the full life stages of individuals within a population and across species. 
\end{enumerate}
\item Fundamental trade-off in life history that climate change is (most?) rapidly altering: growing season length (mean) versus risk (variability often)
\begin{enumerate}
\item Frost risk
\item Drought risk 
\end{enumerate}
\item What extremes? (could move to box 2?)
\begin{enumerate}
\item Heat: just the max (right?)
\item Cold
\begin{enumerate}
\item Max (hardiness at max period) 
\item Transition periods
\end{enumerate}
\item Drought
\item Others?
\end{enumerate}
\item Why biological hazards should be easier to predict than climatic hazards (transition to next section with this? Or find it new home?)
\begin{enumerate}
\item Extremes in climate science are defined statistically usually -- they are rare by definition which makes them tricky
\item But we're not talking about that, we're talking about climate hazards, which depend on biological limits, often thermal limits
\end{enumerate}
\end{enumerate}
\item When extremes matter (need Fig)
\begin{enumerate}
\item Variance shifts in certain biological periods matter a lot more than others: Likely windows for hazards ...
\begin{enumerate}
\item Transitional climatic periods (spring/fall) and hottest summer months
\item But also depend a lot on phenology 
\end{enumerate}
\end{enumerate}
\item PHENOFIT case study
\begin{enumerate}
\item Do means versus variability shifts alone lead to increases or decreases in fitness? Or is is messier?
\item Additive effects... 
\end{enumerate}
\item From models to forecasting (\emph{this section needs work})
\begin{enumerate}
\item Need more fitness data
\item Need more data on events and their impacts (crops?)
\item Need better molecular studies of mean versus variability 
\item Need to bridge the observational/experimental gap for heat extremes (especially)
\end{enumerate}
\end{enumerate}

{\bf Boxes?}
\begin{enumerate}
\item Box 1: Why we're bad at predicting extremes
\begin{enumerate}
\item We may often have poor intuition about what is shifting ... How much are means versus variability shifting? % Empirical reality of shifts in variance
\begin{enumerate}
\item Info from the literature
\begin{enumerate}
\item Basics from IPCC?
\item Refs from Ben?
\item Extremes versus variability
\end{enumerate}
\item Lots of variance patterns in temperature are narrowing: 
\begin{enumerate}
\item Daily temperature ranges
\item Elevation, latitude
\end{enumerate}
\end{enumerate}
\item Case study in Europe
\begin{enumerate}
\item Variability across space (sites)
\item Shifts since 1950 by month 
\item Change over time versus sites: Variability across space decreasing?
\item Projections
\end{enumerate}
\end{enumerate}
\item Box 2: High versus low temperatures
\end{enumerate}

\section{Reference notes}
\citet{Ruel:1999mp} covers Jensen's Inequality; mainly just that you can get the wrong mean if you don't take into account variance; and you can predict useful stuff if you know the variance frequency and some stuff.

\citet{vasseur2014} is again Jensen's Inequality but this time with climate change---looks at mean vs. skew in variance studies of thermal thresholds (I think); focuses on increase in variance. 

\citet{Bennett2021} Show that cold temperature thresholds evolve more than high temperature (so this goes with limited acclimation also for high temperatures). 

{\bf Should review:} Buckley \& Kingsolver AREES

\newpage
\section{Manuscript text} 

{\bf Title:} Seasonal pressure points of climate change

\begin{abstract}
Climate change is reshaping growing seasons globally with major impacts on natural and agricultural ecosystems. Yet we are uncertain exactly how, where, and when impacts will be most pronounced. We show how fundamental life history theory and physiology can help identify the pressure points of climate change---seasonal periods when shifts in climate interact with development to lower growth, reproduction or survival. Using an integrated model of the full annual cycle of plant growth, reproduction and survival (PHENOFIT), we will compare the impacts of future warming versus shifts in frost events on the fitness of three tree species (\emph{Fagus sylvatica, Pinus sylvestris, Quercus robur}). This framework will help identify the challenges and opportunities in adapting to climate change across European forests. 
\end{abstract}

\end{document}

\begin{enumerate}
\item 
\end{enumerate}

% text from fulbright2021.tex 

Climate change is reshaping growing seasons globally with major impacts on natural, agricultural and other ecosystems\cite{IPCC:2014sm}. Decades of research show the most prevalent biological impact is shifts in the timing of recurring life cycle events---known as phenology\cite{Root:2005fr,camille2015,menzel2020}. With warming many plants leafout earlier in the spring, and the timing of reproduction has changed for diverse plant and animal taxa. 

Such phenological changes may have cascading impacts from populations to ecosystems, but exactly how, where, and when the impacts will be most pronounced is uncertain. This is partly due to climate change itself: while average temperatures are higher in most regions, warming is uneven across seasons, years and space. Further, changes in extreme events, such as frosts or heatwaves, are harder to predict but often have large effects\cite{ipcc2013,ipccextreme2012}. 

Predicting points of impact also critically requires understanding when shifts in phenology trigger changes in growth, reproduction or survival\cite{Johansson2012,Lane2012}. This topic is central to life history theory\cite{memegan2021}, but climate change research in this area has generally focused on singular events: for example, reduced growth when earlier leafout is followed by a freeze event\cite{zohner2020pnas,cat2021pep,cat2021exp}, or reproductive declines when the timing of bird nesting becomes mismatched from its food resource\cite{kharouba2018,vissergienapp2019}. Such results show how phenology can affect organisms at a singular point in time, but they generally ignore the rest of the annual cycle\cite{Post:2008em,dan2021nph}, making it difficult to know if losses at one point may be offset by gains at other points. 

Beyond the growing season, phenological studies almost always ignore the remainder of the annual cycle\cite{chuinearees,chuine2016}. Specifically, they ignore periods of `rest' or dormancy, even though such periods have major implications for survival\cite{frostbook,Zanne2914,chang2021}. For many plants, dormancy coincides with acclimation to withstand cold temperature (cold hardiness) and thus is critical to winter survival. While hardiness and phenology have generally been studied in isolation, recent work suggests the two are fundamentally linked\cite{chang2021,kovaleskipreprint}. 

To understand how climate change impacts fitness requires models that connect phenological events with models of growth, reproduction and winter survival. Such models would capture the shifting seasonality of climate change by building climate response curves for the sequence of developmental events in an organism's annual cycle  (Fig. \ref{fig}). In the dormant season these models thus predict the maximum cold a plant can withstand, and in the growing season they predict accelerated development that triggers earlier events as temperatures warm. Accelerated development, however, occurs only until some optimum temperature, with higher temperatures triggering damage. Exactly which temperatures are too cold, optimal, or too high, varies across species and genotypes (variation within species)---complexity that is likely critical to accurate predictions\cite{tansley,chuinearees}. % As such, these models will predict when climate will impact growth, reproduction, and survival, and allow direct comparisons of which events have the largest impacts and the balance of losses and gains from climate change.
% For many plants this sequence is: cold hardiness and dormancy, start of growth (e.g., leafout), then reproduction via flowering, fruiting and seed ripening

Working with collaborators from France, I propose to develop an integrated model of the full annual cycle of plant growth, reproduction and survival to identify the pressure points of climate change---seasonal periods when shifts in climate interact with developmental shifts in hardiness or phenology to lower growth, reproduction or survival, using winegrapes as my model system. Winegrapes are sensitive to temperature and often considered a sentinel of climate change\cite{Chuine:2004fk}. I have studied them for over ten years, using the depth of physiological knowledge and extensive data resources of winegrapes to develop cutting-edge models that help answer  critical questions of how to adapt agriculture to climate change. In particular, I have highlighted how different varieties\cite{moralescastilla,merrill2020}---distinct genotypes, such as Pinot noir or Cabernet-Sauvignon---can vary tremendously in their phenology, and resilience to drought and heat extremes\cite{Wolkovich2017}. For this project, I aim to answer: 
\begin{enumerate}
\item What has been the impact of recent climate change on growth, reproduction (yield) and survival across both the growing season and dormancy period? 
\item How do these impacts vary across seasons, space and different winegrape varieties?  I expect that the impact of freeze events may be more extreme for certain varieties and at the poleward range edges; and that the relative importance of losses from freeze events relative to losses from heat extremes later in the season will also vary predictably. 
\item What are the projected impacts of climate change on growth, reproduction and survival in the future, given continued warming? I am especially interested in both potential pressure points in the future, and whether there are obvious areas of leverage---for example, certain varieties or parts of the season that may allow greater resilience with climate change. 
\end{enumerate}
\vspace{-1ex}
\begin{figure}
  \begin{center}
\includegraphics[width=1\textwidth]{figures/sabbfig.png}
  \end{center}
  \caption{Climate change alters the depth and timing of cold hardiness (the maximum temperature tissues can withstand before death, shown here as `lower lethal temperature') and critical growing season events, reshaping how hardiness and phenology interact with climate to determine growth, reproduction and survival. (A) a hypothetical pre-climate year includes several freeze events early in the year, but temperatures are never beyond the plant's maximum during major events; in contrast (B), a year with warming has fewer freeze events, earlier and accelerated budburst and flowering, but slower ripening and the potential loss of fruit due to heat extremes. }
\label{fig}
\end{figure}
\vspace{-1ex}
\emph{Research plan}: Given their importance to agricultural and climate forecasts, varying models of the complete plant life cycle have been developed\cite{Chuine:2001ab}, especially for agricultural crops\cite{bowen2015}. Current models, however, often do not consider hardiness, rarely consider multiple varieties and are often so extensively parameterized (for example, including temperature response curves within larger models of carbon and nutrient pools in the soil and plant) that it is difficult to robustly estimate temperature responses. Many models simply set the upper limit for many plant events at 40$\degree$C, even though this limit varies across species, populations and phenological events\cite{parent2012}.

I plan to leverage my existing expertise in phenological models, data synthesis, hierarchical Bayesian approaches\cite{flynn2018,ospreebbms} and my developing research in plant hardiness\cite{cat2021pep,cat2021exp,cat2019} to build a joint model of developmental sequences for winegrapes---one of the world's most economically important crops\cite{faostat}. 
% plants focused on temperature as dominant controller. By focusing on temperature responses and using cutting-edge Bayesian approaches\cite{BDA,Carpenter:2017stan}, I will be able to estimate all parameters together and thus more accurately estimate temperatures responses and their uncertainity. 

Models of dormancy will be focused mainly on cold hardiness, the critical fitness component during the rest period. I expect to draw on models from pharmacokinetics/pharmacodynamics\cite{pkpd2015}. My lab currently has a hardiness model based on a basic dose-response curve, with the `dose' being cold temperatures that induce cold hardiness up to some maximum threshold; I will expand on this fundamental advance to model the onset and release of cold hardiness. These two phases are critical as plants are highly vulnerable in these transition periods. % Cold hardiness release will then be linked to the onset of development leading to budburst. 

To model growing season events (budburst, flowering, fruit ripening) I will borrow from elements of basic survival models, which typically model the event of death as dependent on some underlying process. Such models, however, are ideal for modeling event processes as they can at once determine the shape of the temperature response curve (i.e., what temperatures are too low, optimal or too high) and the required threshold to transition between events. To both leverage the most data and to compare across varieties, I will use Bayesian hierarchical modeling\cite{BDA,Carpenter:2017stan} to include at least 10 different winegrape varieties (for each variety I will obtain unique estimates of temperature responses and will also estimate an overall mean response across varieties). 

Robustly fitting these models will require data of cold hardiness (50\% LTe, i.e. lower lethal temperature), phenological data---all with matched climate data. Winegrapes uniquely provide excellent data for both hardiness and phenology, and my lab has worked to gather data over the last seven years, which I will leverage. These data include a mix of long-term observations, including data stretching back centuries for Pinot noir in Burgundy, and data of many varieties from global winegrowing regions, including across France (e.g., Alsace, Bordeaux, Burgundy, Champagne), California (Napa, Sonoma, Central Valley) and the Okanagan Valley, Canada, that may lab has collected. Collaborators at INRAE Avignon, Montpellier SupAgro and INRAE Domaine de Vassal will additionally provide more data as needed. 

Once the models are developed I will validate them using hindcasting, then predict potential gains and losses in growth, reproduction and survival to 1960 across winegrowing regions in Europe and North America, using gridded daily climate datasets to  (E-OBS for Europe, and DayMet for North America; shape files for all winegrowing regions provided by my collaborator, Dr. Ignacio Morales-Castilla). I plan to focus on estimating: loss to frost across the winter season with a special focus on losses in the early fall and late spring, length of growing season, as a proxy for growth, and potential losses of flowers or fruit due to heat extremes (with extremes defined as days at least 0.5$\degree$C above a variety's estimated maximum temperature for that developmental phase). Using scenarios from CMIP6 (global models of future climate)\cite{eyring2016,riahiSSP2017,cookcmip} I will then generate forecasts of the future of winegrowing regions integrating variety diversity. \\

\emph{Expected outcomes \& communication pathways:} My goal is to generate new understanding of the connectedness of hardiness and the growth cycle of winegrapes. I believe this approach could revolutionize how we think about impacts of climate change on crops and help unify research on hardiness and growing season phenology and fitness. Additionally I expect that by joint modeling the full annual cycle, I will highlight key areas of uncertainty in our understanding of plant temperature response, knowledge my lab and collaborators can use to tailor future experiments and studies. % Can add my work on forest trees here? % The method can be adapted other perennial crops and to wild vine, shrub and tree species. 

The combined model and output for winegrapes across both Europe and North America and a suite of varieties will highlight both the opportunities and pressure points for winegrapes as a crop with climate change. I expect many areas may experience increased potential losses from shifts in frost regimes or heat extremes, but that diversity in responses across varieties may allow from adaptation options for growers. Further, I expect the holistic view will help highlight the complexity of climate change. For example, the northern edges of winegrowing today may see benefits of increased growth, but potentially also losses depending on variety or the season. 